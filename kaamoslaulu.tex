\documentclass[12pt,a4paper]{article}
\input{common.inc}

\begin{document}
\thispagestyle{empty}
\begin{mysong}{Kaamoslaulu}{0}

\SBIntro{\Ch{Am}{\SBLyricNoteFont Maken soolo} \Ch{Em}{}}

\begin{SBVerse}
  \Ch{Am}{Laittaisko} narun kaulaan, \Ch{Em}{vai} pyörimään?
  \Ch{Am}{Tallentamaan} elämää \Ch{Em}{elettyä,}

  hyvää ja \Ch{D}{pahaa,} likaista \Ch{Am}{rahaa}, liian vähän
  \Ch{Em}{rakkautta.}

  \Ch{Am}{Täällä} on kaunista \Ch{Em}{kaikesta} huolimatta. \Ch{Am}{Nouskaa}
  ja nostakaa \Ch{Em}{kätenne} ilmaan.

  Vienosti \Ch{D}{hymyillään,} ollaan hetki \Ch{Am}{hereillä,} eikä mitään
  \Ch{Em}{hävitä.}
\end{SBVerse}

\begin{SBChorus}
  \Ch{Em}{Tää} on se kaamoslaulu, \Ch{A}{joka} laitetaan
  \Chr{C}{levylauta}\Ch{Bm}{selle} pyöri\Ch{Em}{mään.}

  \Ch{Em}{Tää} on se kaamospäivä, \Ch{A}{joka} kestää \Ch{C}{kolme}
  minuut\Ch{Bm}{tia} korkein\Ch{Em}{taan.}
\end{SBChorus}

\begin{SBVerse}
  \Ch{Am}{Verkkokalvoilla} \Ch{Em}{sahalaitoja.} vai \Ch{Am}{onko} se unta,
  \Ch{Em}{mustaa} lunta,

  joka \Ch{D}{sataa} puoli\Ch{Am}{väliin,} ja on taas \Ch{Em}{poissa?}

  \Ch{Am}{Niin} kuin meistä jokainen, \Ch{Em}{sinä} ja hän: \Ch{Am}{eletty} tää
  elämä, \Ch{Em}{tehty} liian vähän.

  Kun tullaan \Ch{D}{aamuun,} joka ei \Ch{Am}{ala}, eikä se \Ch{Em}{lopu.}
\end{SBVerse}

\begin{SBChorus}
  \Ch{Em}{Tää} on se kaamoslaulu, \Ch{A}{joka} laitetaan
  \Chr{C}{levylauta}\Ch{Bm}{selle} pyöri\Ch{Em}{mään.}

  \Ch{Em}{Tää} on se kaamospäivä, \Ch{A}{joka} kestää \Ch{C}{kolme}
  minuut\Ch{Bm}{tia} korkein\Ch{Em}{taan.}
\end{SBChorus}

\SBBridge{
  \Ch{Em}{\SBLyricNoteFont Maken soolo} \Ch{A}{} \Ch{C}{} \Ch{Bm}{} \Ch{Em}{}
  \Ch{A}{} \Ch{C}{} \Ch{Bm}{} \Ch{Em}{}
}

\begin{SBChorus}
  \Ch{Em}{Tää} on se kaamoslaulu, \Ch{A}{joka} laitetaan
  \Chr{C}{levylauta}\Ch{Bm}{selle} pyöri\Ch{Em}{mään.}

  \Ch{Em}{Tää} on se kaamospäivä, \Ch{A}{joka} kestää \Ch{C}{kolme}
  minuut\Ch{Bm}{tia} korkein\Ch{Em}{taan.}
\end{SBChorus}

\SBEnd{\Chr{Em}{} \Ch{A}{} \Ch{C}{} \Ch{Bm}{} \Ch{Em}{}}

{\SBLyricNoteFont Johan: P\textsubscript{2}23~Or:Spyral, scene~1+2,
kertosäkeessä mod$\uparrow$.}

\end{mysong}
\end{document}
